\section{Summary and discussion}

To conclude, electrons in liquid helium form an interesting quantum mechanical system and can be studied through experiments. The goal of experimenting with electrons was also to design future experiments for studying exotic ions. These electron bubbles can be cavitated by applying a certain negative pressure. We found that there are two different thresholds for cavitation in electron bubbles. Each has a different mechanism which was studied in detail. The experiments also suggested a new event called very rare event of which not much is known. While studying the variation of critical static pressure with temperature, we were motivated to perform calculations for non linear propagation of sound in liquid helium.

The calculations for propagation of sound were quite successful in modeling the system. All the basic requirements were met and we recorded the correct pressure oscillation. We saw that the relation between critical static pressure and temperature is indeed nonlinear. Thus this simulation would help in studying exotic ions in the future.


\section{Future directions}
In order to understand the structure and energy levels of these exotic ions, experiments involving quantized vortices can be performed. It is possible to trap exotic ions inside quantum vortices. \cite{Milliken1982} Once trapped, the exotic ion can be excited with a laser. \cite{DEGROOT1985445} Finding the critical wavelength- where the exotic ion barely escapes the vortex- would help in determining the energy levels of the exotic ion.

Recent experiments involving calculating the mobility of charged ions in liquid helium suggests the existence of positively charged ions in liquid helium. It is possible that these ions could be clusters of helium molecules around a positive helium ion. In order to see if that is possible, we ran molecular dynamics simulations. The formation of the cluster was identified using the coefficient of self-diffusion. If there is a cluster formed, the coefficient of the diffusion for atoms near the charged molecule should be much smaller than for molecules far from it. The ultimate goal is to start with a positively charged helium molecule and neutral helium molecule in a box. Then we would let them interact for a while and see if they clump together to form a cluster. Later we can attempt to model the structure of this cluster. Finally, we need to make sure that the simulation is modeling the system correctly and make detailed calculations involving diffusion coefficients.

It was concluded that it is difficult to investigate the displacement-static pressure relationship for the thickness mode by means of simulation. This is because the sound field produced by the thickness mode is very complex. It was found that the transducer vibrates in a complex manner when driven with high frequency. These vibrations cannot be represented by a single equation and thus more sophisticated simulations need to be performed. It is also possible that if the pressure swing is big enough, the numerical simulation might explode at the focus. This might be due to a shock formation. A possible solution is to use the WENO scheme. \cite{Appert2003}